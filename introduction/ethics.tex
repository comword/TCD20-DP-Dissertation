\section{Research ethics}
\label{sec:Research ethics}
%2 pages: 1 for concerns, 1 for solutions
According to the requirements of research ethics, any research involving human participation must have reviewed and got an ethical approval by the Research Ethics Committee.
Obviously, this research is human activity, whose videos also need to be captured from participants as training or validation data set, so it requires some ethical discussions before conducting research.

The ethics committee raised privacy concerns during the review process that people in the videos have facial biometric information which directly identifies participants, and inadvertent capture of sensitive information from surroundings or computer screen might menace examinees' privacy.

In terms of privacy, \citet{coghlan2020good} philosophically describe the ethical rationality of online proctoring technologies, which highlights ``academic integrity, fairness, non-maleficence, transparency, privacy, respect for autonomy, liberty, and trust''.
Finally, they conclude that the online proctoring function requires a precedent balancing between concerns and possible benefits.
Educational institutions also have the duty to take ethical considerations should they decide to adopt the technologies.

\citet{bozkurt2020education} mention that laws and regulations should strive for online data protection and privacy, such as the General Data Protection Regulation (GDPR) in Europe, which gives people more control over their data and regulates enterprises in storing and transmitting user data.
In this way, with the continuous advancement of the security measures, invigilation and monitoring strategies in online exams, laws and regulations will protect learners from the side-effects of invigilation and monitoring related practices.

In order to alleviate the privacy concerns, this research uses anonymisation algorithms in the data preprocessing to ensure that participants in the original video will be anonymised.
This process will minimise the presence of any unnecessary material that might enable the identification of an individual, for example, to cover the detected face area with extracted facial landmarks.
The detailed processing steps and algorithm details will be explained in chapter \ref{chap:Design}.