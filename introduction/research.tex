\section{Research aims}
\label{sec:Research aims}
%1 page
After clarifying the research background, the research question of this dissertation is summarised as the following:

\begin{quote}
    Is it possible to understand sequential human actions from an on-device camera video input stream with deep learning technologies to classify the series of actions of the exam attendee to approved or prohibited behaviours with high accuracy?
\end{quote}

In order to answer the research question, this research aims to achieve the following objectives through the corresponding methods.

\begin{longtable}{>{\hspace*{-0.3cm}$\bullet$\hspace*{0.2cm}}p{.4\textwidth}p{.56\textwidth}}
\textbf{Research objectives} & \textbf{Method to achieve objectives} \\ \hline
To do the literature review & By surveying the state-of-the-art technologies of deep-learning-based human action recognition \\ \hline
To have an available data set & By obtaining a suitable image and video data set and labelling them for training the model \\ \hline
To design a deep learning model & By designing a deep learning model \\ \hline
To code and train the deep model & By training the model and fine-tuning hyper-parameters \\ \hline
To optimise the model for mobile & By developing a working mobile app that equips the model \\ 
\end{longtable}

After completing the above objectives, this research will also evaluate the project outcomes, especially for model performance and other following aspects.

\begin{itemize}
    \item Model accuracy, precision and other evaluation metrics on the classification task.
    \item Balanced performance between accuracy and efficiency of the deep learning model.
    \item User experiences of the overall system.
\end{itemize}
