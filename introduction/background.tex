\section{Research background}
\label{sec:Research background}
%1.5 pages
%COVID-19 impacts education
COVID-19 has brought many challenges to higher education.
\citet{marinoni2020impact} reiterate the facts released by UNESCO regarding the significant impact of the pandemic on education in countries around the world.
For example, there are approximately 130 million students, accounting for 89.4\% of total enrolled learners, whose academic career has been vitiated because of the impact of pervading virus caused school closures.

To better respond to the crisis and allay public concerns caused by the epidemic, the International Association of Universities (IAU) launches a survey of the impact of COVID-19 on higher education.
\citet{marinoni2020impact} present the survey results showing that COVID-19 has inevitably affected many processes in education, including teaching \& learning, researching, and assessment.

In teaching and learning, most classroom lectures have been substituted by distance teaching and learning, which may not be difficult for professors and students who are familiar with computers and the Internet.
On the other hand, experiments and research that require the use of shared professional equipment are much more severely affected.
The survey shows that 80\% of researchers have reported that their research progress is decelerating and moving at a creep.

However, as for assessment and examination, institutions have no way to perpetuating the traditional assessment methods, which is indicated by \citet{clark2020testing} as one of the most important challenges for students.
``Learning assessment and examination approaches will be reviewed, and institutions may choose to invest further in technical infrastructures'', said by \citet{marinoni2020impact} points out the fact.
Although some universities shift from closed-book exams to pure continuous assessments immediately, other educational institutions are craving new technologies to help them get out of the trouble.

As a result, the motivation of this research is to develop a practical application in human activity recognition, and to provide a way for educational institutions to hold exams and reduce labor costs of distance invigilation.

%Deep learning technologies are widely used, applications
In recent years, ubiquitous influences of deep learning methods have brought tremendous advancement to many computer science fields and individual's daily life. 
In many fields, such as computer vision and natural language processing, deep learning approaches achieved higher performance than traditional algorithms.
Therefore, this research is to explore deep learning techniques to solve automatic exam invigilation problems in the assessment process of distance education.

%Deep learning technologies are research hot-spot
The research of deep learning is also a hot topic in contemporary computer science fields.
The design concepts between different deep models are also mutually influential.
Especially while convolutional neural network (CNN) is in the ascendant among the fields of computer vision, transformer model composed of self-attention structures has achieved state-of-the-art results in many natural language processing tasks.
The transformer model gradually replaced the recurrent neural network (RRN) with sequential computing restriction and long-term memory loss issues.
Innovative design concepts from the transformer model are also carried forward in the fields of computer vision.
Many recent studies, such as Data-efficient image Transformers (DeiT) and Shifted window (Swin) transformers, have improved widely-adopted CNN-only models such as VGG and ResNet by introducing the self-attention structures to achieve better performance.

This research is based on previous studies in deep learning and deep model mobilisation.
Currently, there are some existing models for human posture analysis and detection.
But they use static imagery as input and key points of the body as output, and they are not optimised for mobile devices with limited computing power.
Beside, the personal computer hardware, the bridge between the software system and the physical space, is not trusted.
However, a trusted software environment can be ensured on mobile phones with hardware attestation functions provided by operating system and device manufactures.
