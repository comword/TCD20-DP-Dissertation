\section{Deep model design}
\label{sec:Deep model design}
Since deep networks have the duties of feature extraction, deep learning methods no longer require manual feature engineering.
But it is still necessary to consider that different deep networks have inductive biases (preferences) for different feature types.
The literature review section \ref{sec:Deep models detail} has reviewed models detail and instinct properties.
For example, CNN has properties of translation invariance and is suitable for imagery features; Transformers has properties of long-range dependency modelling and adapts to time-series features.

The model design procedure includes not only the network architecture but also the data loader.
The latter is responsible for reading the result from the previous data pre-processing procedure and transforming them into the tensor shapes required by the model input layer.
Because the data loader is the foremost part of the model design, this section will begin with the detail of loading data.
Then, the deep network architecture for the research goal will be detailed, as well as the output layer for generating probability results.

\subsection{Data loader and input layer}
The data loader is a binding bridge between the processed data set and the model input layer, which converts the data format from the former into information that the model's input layer can accept.
As discussed in pre-processing section \ref{sec:Data preprocessing}, the results from the previous procedure are individual video frames saved in picture format.
Thus, any picture library in Python can read the inputs into memory, such as OpenCV, which was also previously used in data pre-processing.

Table \ref{tab:Model input tensors} designs input tensors of the model.


\begin{table}[!ht]
\renewcommand{\arraystretch}{1.6}
\begin{tabularx}{\textwidth}{|c|c|c|X|}
\hline
Tensor ID & Name           & Shape                    & Description                                            \\ \hline
0         & Images   & {[}1, 3, 16, 224, 224{]} & {[}Batch size, RGB channels, Frames, Height, Weight{]} \\ \hline
1         & Positions & {[}1, 16{]}              & {[}Batch size, Frame position{]}                       \\ \hline
\end{tabularx}
\caption{Model input tensors}
\label{tab:Model input tensors}
\end{table}

\subsection{EfficientNet spatial backbone}
Minimal EfficientNet backbone pretrained on ImageNet data set to do spatial feature extraction from frames.
subsection \ref{subsec:Evolution from MobileNet to EfficientNet}

\subsection{Longformer temporal backbone}
Longformer with global and local sliding window attention pattern as a temporal encoder.
subsection \ref{subsec:Optimisation of Transformer Networks}

\subsection{Output layer and data format}
the MLP classification header with 2 fully connected dense layers

\begin{table}[!ht]
\renewcommand{\arraystretch}{1.4}
\definecolor{caribbeangreen}{rgb}{0.0, 0.8, 0.6}
\definecolor{amber}{rgb}{1.0, 0.75, 0.0}
\definecolor{carminered}{rgb}{1.0, 0.0, 0.22}
\newcommand{\green}{\color{caribbeangreen} Green}
\newcommand{\amber}{\color{amber} Amber}
\newcommand{\red}{\color{carminered} Alert}
\begin{tabularx}{\textwidth}{|c|c|c|X|}
\hline
ID & Activity    & Alert level & Description                      \\ \hline
0           & Unknown     & \amber & The model cannot predict the current activity that may be a non-exam activity \\ \hline
1           & Look screen & \green & The student is looking at the screen \\ \hline
2           & Look down   & \red   & The student is looking under the table, which maybe using the phone \\ \hline
3           & Look side   & \amber & The student is looking outside the computer screen, maybe reading a book on the table \\ \hline
4           & Look back   & \amber & The student turned and is looking behind, maybe hiding something \\ \hline
5           & Leave       & \amber & The student is out of the camera range \\ \hline
6           & Speaking    & \red   & The student has obvious lip movement and is talking to others \\ \hline
7           & Look up     & \green & The student looks up at the ceiling, may be relaxing or thinking a difficult question \\ \hline
8           & Use phone   & \red   & The student are obviously using a mobile phone \\ \hline
9           & Scratching  & \green & The student is scratching the head or face and may be thinking a difficult question \\ \hline
10          & Drinking    & \green & The student is drinking water \\ \hline
11          & Typing      & \green & The student is typing on the keyboard and answering exam questions \\ \hline
12 -- 15    & Unused      & --     & Allow for adding more activities \\ \hline
\end{tabularx}
\caption{Model output for exam activity categories}
\label{tab:Model output}
\end{table}
