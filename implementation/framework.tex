\section{Framework selection}
\label{sec:Framework selection}
Because the scope of this research is relatively wide, covering data collection, model implementation, and mobile app implementation, many frameworks and libraries are used in implementing the system.
Obviously, depending on target platforms, the programming language used in each part is also different.
The next subsections will introduce both programming languages and frameworks used to develop the corresponding subsystem.

\subsection{Web development} %1 page
Web technology is developing and advancing rapidly, which is the reason for the extensive use of web technology in this study.
From the earliest era of directly programming HTML, CSS, and JavaScript, the web technology developed through templated server-side-rendering, such as PHP Hypertext Preprocessor or Jakarta Server Pages (JSP).
Recently, web development has become more structured and engineered by using client-side rendering framework for front-end, such as React, Vue or Angular and microservice framework for back-end.
Cross-platform UI frameworks such as Electron, React-Native, and Weex make web technology the preferred solution for developing cross-platform applications on the client side.

Web technologies have been majorly used both in the web-based data collection app and the mobile app user interface/user experience(UI/UX) design.
TypeScript, a strongly typed programming language building upon JavaScript, is used for the web front-end development of the data collection app.
As for the back-end development, this study uses Golang for a better gRPC experience.

Node.js is used as the front-end development environment in this study.
It was initiated by \citet{nodejs2021} in mid-2009, aimed to create a high-efficient JavaScript runtime.
Node.js is built on Google Chrome's V8 JavaScript engine, one of the most efficient JavaScript engines and is widely adopted in front-end development because of its powerful package manager and excellent ecosystem for rich frameworks and libraries.

The data collection app requires a user interface, implement data flows, thus, using a mature front-end framework is the best choice.
Although many famous frameworks to choose from, such as React, Vue and Angular, I finally decided to use React, a declarative, efficient, and flexible JavaScript library for building user interfaces.
Because it is less complex compared to the other alternatives and it is being supported and maintained by \citet{react2021}.
It also has a good ecosystem and many successful apps are developed based on it, which is beneficial to explore in the project.

As for Redux, it is a predictable state container for JavaScript applications also a good companion with React, developed by \citet{redux2021}.
It helps create applications that behave consistently and predictably with complicated states, and keeps the states and connects each state with views in React.
Any action that causes change to the state is reversible with a time-travelling debugger, which provides a great developing experience.

\subsection{Deep learning model development} %.5 page
In the field of deep learning, TensorFlow and PyTorch are two famous and widely used frameworks.
They both support the use of GPU for acceleration in the model training process, but there are still many differences in programming paradigm and application areas.

%PyTorch
PyTorch is an open source Python library widely used in the academic.
It was published in early-2017 by the Facebook's AI Research Lab.
In 2019, \citet{steiner2019pytorch} pointed out that Caffe, TensorFlow (version 1.x at that time), and Theano all construct a static computation dataflow graph to achieve high performance at the cost of usability, debugging, and flexibility, but PyTorch is a framework with dynamic eager execution, which enables high usability without sacrificing performance.
Another study by \citet{florencio2019performance} in the same year conducted a detailed comparative study on the performance of Tensorflow and PyTorch.
After data analysis, they concluded that TensorFlow has higher GPU utilisation, but the PyTorch shows better overall performance.

%Tensorflow
As for Tensorflow, \citet{abadi2015tensorflow} proposed that it is a deep learning framework developed by the Google Brain team in 2015.
It has a relatively long history in the deep learning field and is widely adopted in enterprises and industries.
In TensorFlow 1.x version, the computation dataflow graph is statically defined through \textit{tf.session} and \textit{tf.placeholder} before running the model.
This old way sacrifices usability for performance and makes it impossible to set breakpoints to view data during the computation process, which causes great inconvenience to the debugging progress.

However, \citet{singh2020introduction}

\citet{luo2020comparison}

\subsection{Android application development} %1 page
% TFLite
\citet{singh2020mobile}

\citet{fadlilah2021development}

React Native \citet{eisenman2015learning}