\chapter{Conclusion}
\label{chap:Conclusion}
% 1 page discussion and summary, 1 page possible future works
The previous chapters have outlined each part of the system from the design and implementation aspects.
The evaluation chapter has shown and critically analysed the experiment results of the deep model evaluation and the mobile application performance in multiple evaluation metrics.
This chapter will summarise the evaluation results and draw research conclusions regarding the research question and propose possible future work in this field.

% Summary
The evaluation results have verified the research question that it is possible to classify the exam attendee's activities into approved or prohibited behaviours using an on-device camera video input stream.
Through my enormous efforts contributed to this research project and dissertation, I have completed this study and summarised my contributions to the following points.

\begin{itemize}[leftmargin=1em]
    \setlength\itemsep{-.5em}
    \item This study spots an urgent need during the COVID-19 epidemic on a human action recognition task -- the remote exam invigilation.
    \item This research surveys and reviews previous research on the design details and mobile optimisation insights of many related deep models.
    \item This study designs and implements a web-based app to collect labelled video data from research participants.
    \item This study designs and implements a deep model to accomplish the research goal by applying the deep-learning-based video understanding technique.
    \item This study ports and optimises the proposed deep model to Android mobile devices, which validates the feasibility of running the app equipped with the deep model.
    \item This study evaluates the proposed deep model in multiple metrics and conducts performance experiments on two Android phone models, OnePlus 5 and OnePlus 8.
\end{itemize}

The solution proposed in this research only solves remote exam invigilation in physical space, ensured exam attendees only interact with the device used to sit the exam.
The limitation of the solution lies in the inability of controlling software on the personal device.
For example, a dishonest student can use remote desktop software to enable impersonation, letting others cheat for the exam through the Internet. 
Because the dishonest student does not do any prohibited activities in the physical space, the solution does not apply to this situation.

% Future works
As for future work, the possible future research directions in the field of remote examination invigilation are considered based on the project outcomes shown above, which is mainly four directions -- data set volume, deep model optimisation for both model architecture and mobile devices, integration with online exam platforms and areas of security and privacy.
The first two of these research directions are pushing forward the deep model performance.
And the last two directions are considering more user experience engineering or related to security and privacy, which is beyond the research scope of deep learning.

In detail, the first enhancement direction is generally effective to any deep models, that is, to increase the total data set volume.
The evaluation result unveils that the deep model still has the potential and capacity to train on larger data sets.
Future research can integrate data collection into the mobile app and continuously improve the model to adapt to various video angles.

The second way of improving the system performance is to further optimise the deep model on the accuracy and computational complexity.
Factorised model variants used in Video Vision Transformer network proposed by \citet{arnab2021vivit} pointed out a method to further enhance the performance of deep models designed for video understanding tasks.
The idea behind the factorisation is to decompose the monolithic structure of the spatial encoder first and then the temporal encoder into more and smaller encoders distributed in layers.
More communications and fusions between spatial and temporal data facilitates the model to capture richer features across spatiotemporal space.

In 2021 August, as this dissertation is about to be completed, \citet{chen2021mobileformer} proposed a Mobile-Former that bridges MobileNet and Transformer.
They creatively proposed a parallel design of MobileNet and Transformer, interspersed with the bidirectional fusion of ``MobileNet at local processing and Transformer at global interaction'', which takes the advantages of both convolution and Transformer to achieve \textit{mutuo auxilio}.
Future works may use the Mobile-Former to achieve better computational efficiency and more representation power.
Besides, more deep learning frameworks are worth trying out in the model implementation.
For example, \citet{reiss2020pytorch} presented that PyTorch used TorchScript to enable running on mobile devices and supported NNAPI by the end of 2020.
On this basis, A deep learning library for video understanding research\footnote{PyTorchVideo: \url{https://github.com/facebookresearch/pytorchvideo}} is published during this research period and is worth trying in future work.
