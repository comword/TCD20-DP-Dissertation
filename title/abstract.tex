\newpage
\vspace*{1cm}
\begin{center}
\linespread{1.3}
\selectfont
\huge \bfseries \thesistitle\\[1cm]
\end{center}
\begin{center}
\authorname\\
\href{mailto:geto@tcd.ie}{geto@tcd.ie}\\
University of Dublin, Trinity College, 2021
\end{center}
\vspace*{1cm}
\phantomsection
\addcontentsline{toc}{chapter}{Abstract}
\begin{abstract}
\begin{changemargin}{1cm}{1cm}
The COVID-19 pandemic has prevented students from congregating to take traditional in-person exams, shifting the attention of pedagogical institutions to online exam systems accessed remotely. 
In this research, previous review papers on human action recognition are surveyed to confirm the feasibility of obtaining action features from spatiotemporal data sets such as videos.
This research reviews multiple state-of-the-art deep models and optimisation methods under the premise of achieving a balance between performance, function and resource demand constraints of mobile devices.
Then a deep learning model is developed to classify the behaviour of examinees to a series of activities, approved or prohibited by the exam holder.
The model evaluation fully considers both computational performance and other commonly used metrics in machine learning classification tasks.
Experimental result shows that the proposed model achieved over 90\% accuracy over 12 exam-related action categories with 23.6M parameters, making it possible to run inference smoothly on most decent mobile phones.
Finally, this research concludes model design and optimisation experiences with possible directions for future research in this area.

\vspace{1cm}
\begin{flushleft}
   \let\and\\%
   \textbf{Keywords:} \keywords
\end{flushleft}
\end{changemargin}
\end{abstract}

\newpage