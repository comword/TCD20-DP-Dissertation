\section{Mobile app evaluation}
\label{sec:Mobile app evaluation}
Since this study has the research scope of mobile platforms, experimenting with computation performance evaluation is necessary to show the deep model is optimised to run inference on mobile devices.
The experiment in this research uses two mobile phone models from OnePlus.
The first is the OnePlus 5, a bit old phone released in June 2017, and the second is the OnePlus 8, one of the top phones from last year, unveiled on April 14, 2020.
Table \ref{tab:Experiment results on mobile devices} shows the hardware details of these phones with the experiment results of average frame rate expressed in frames per second (FPS) metric.

\begin{table}[!htbp]
\renewcommand{\arraystretch}{1.4}
\centering
\begin{tabular}{|c|c|c|c|c|c|c|c|}
\hline
\textbf{Device} & \textbf{SoC}   & \textbf{CPU} & \textbf{Process} & \textbf{Accelerator} & \textbf{RAM} & \textbf{FPS} \\ \hline
OnePlus 5       & Snapdragon 835 & \begin{tabular}[c]{@{}c@{}}Kyro 280\\ 2.46 GHz\end{tabular} & 10nm             & \begin{tabular}[c]{@{}c@{}}Adreno 540 GPU\\ Hexagon 682 DSP\end{tabular} & 8GB & \textbf{18} \\ \hline
OnePlus 8       & Snapdragon 865 & \begin{tabular}[c]{@{}c@{}}Kyro 585\\ 2.84 GHz\end{tabular} & 7nm              & \begin{tabular}[c]{@{}c@{}}Adreno 650 GPU\\ Hexagon 698 DSP\end{tabular} & 12GB & \textbf{28} \\ \hline
\end{tabular}
\caption{Experiment results on mobile devices}
\label{tab:Experiment results on mobile devices}
\end{table}

The result from the OnePlus 8 shows that the mobile app runs fast and smoothly on the high-end phone manufactured last year.
Therefore, from 2021 and in the future, there will be no hardware performance barriers for mobile phones to running the deep model implemented in this study.

As for the result from the OnePlus 5, an old mobile phone released four years ago, the frame rate result is lower than the result from the OnePlus 8.
Although users of old mobile phones may feel that the camera display in the app is a little lag and delayed, it does not crash or affect the user operation, UI response effect and the deep model inference because of the well-designed multi-threading architecture in the mobile app.
Besides, exam attendees should not stare at or operate the invigilating mobile phone during the exam, slight lag and delay in the camera display affect nothing in the practical use case.
As a result, a little old phones should be able to run the app as well.

This study originally planned to evaluate more devices and more performance metrics.
However, due to the limited device resources, short research period and foreseeable cumbersome tasks, this part of the work has become one possible direction of future research.
For example, future research may conduct a user study to evaluate the user interface and user logic in the proposed mobile app after the integration with an online exam platform.
Besides, future research may evaluate more metrics on the computational efficiency, such as utilisation rate of CPU, GPU and neural network accelerator.
More in-depth evaluation can further unveil the performance bottleneck in the deep model or program tasks so that future researches can optimise the system much specifically.